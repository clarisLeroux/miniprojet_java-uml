\documentclass[12pt]{article}

\usepackage[utf8]{inputenc}    
\usepackage[T1]{fontenc}
\usepackage[francais]{babel}
\usepackage{amsmath}
\usepackage{amssymb}
\usepackage{mathrsfs}

\begin{document}
 
\title{Chifoumi - Cahier des charges}
\author{M. Duprey \and M. Grimal \and E.Jeanmougin \and C. Leroux}
 
 
 \maketitle
 
 \newpage
 \tableofcontents
\newpage

\section{Règles du jeu}

\begin{itemize}
 \item Les déplacements invalides sont :
\begin{itemize}
 \item[\textbullet] tout déplacement en dehors du plateau 
 \item[\textbullet] tout déplacement sur un pion adverse ne pouvant être pris
\end{itemize}
\item Lorqu'un qu'un pion est pris, il est alors retiré du plateau. Si ce pion était au sommet d'une pile, la totalité de la pile est alors retirée du plateau. 
\item Le nombre de points marqué par un joueur correspond au nombre de pions qu'il a pris à son adversaire.
\item La partie se termine lorsqu'un des joueurs n'a plus de pions ou lorsqu'il ne reste plus qu'une sorte de pions sur le plateau.
\item Le joueur gagnant est celui qui a marqué le plus de points. En cas d'égalité, le joueur qui a commencé perd.
\item Le joueur garde la main tant qu'il n'a pas fourni un déplacement valide
\end{itemize}


\section{Déroulement du programme}
 \begin{itemize}
 \item Au lancement de l'application, le programme
 \begin{itemize}
  \item[\textbullet] crée la grille
  \item[\textbullet] place les pions
  \item[\textbullet] crée et initialise les deux joueurs
 \end{itemize}
 \item En phase de jeu, le programme
 \begin{itemize}
  \item[\textbullet] présente à l'utilisateur l'état du jeu (joueur courant, gagnant s'il existe) et de sa grille (placement des pions)
  \item[\textbullet] fournit à l'utilisateur le moyen de désigner la case jouée
  \end{itemize}
 
  \item affiche une allarme si le déplacement n'est pas valide, sinon change de joueur
  \item actualise ces informations jusqu'à la fin du jeu
 \end{itemize}
\section{version 1 : affichage en console}

\section{version 2 : version graphique}

\section{version 3 : mode réseau}

 
\end{document}
