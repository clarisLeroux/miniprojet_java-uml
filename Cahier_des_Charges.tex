\documentclass{article}

\usepackage[utf8]{inputenc}    
\usepackage[T1]{fontenc}
\usepackage[francais]{babel}
\usepackage{amsmath}
\usepackage{amssymb}
\usepackage{mathrsfs}


\begin{document}
 
\title{Chifoumi - Cahier des charges}
\author{M. Duprey \and M. Grimal \and E.Jeanmougin \and C. Leroux}
 
 
 \maketitle
 
 \newpage
 \tableofcontents
\newpage

\section{Règles du jeu}

\begin{itemize}
 \item C'est un jeu fait pour deux joueurs.
 \item Ce jeu se déroule sur un plateau (une grille de 7*7 cases) en forme de "+".
 \item On dispose sur ce plateau de pions ronds pouvant representer:
 \begin{itemize}
    \item[\textbullet] un puit
    \item[\textbullet] une pierre
    \item[\textbullet] une feuille
    \item[\textbullet] une paire de ciseaux
 \end{itemize}
\item Les pions se déplacent de la façon suivante:
\begin{itemize}
  \item[\textbullet]Déplacement d'un pion sur une case vide
  \item[\textbullet]Déplacement d'un pion sur un pion de sa propre couleur (empilement)
  \item[\textbullet]Déplacement d'un pion sur un pion de l'adversaire pouvant être pris
\end{itemize}
  \item Il est interdit de sortir un pion du plateau pour atteindre la case opposée.
  \item Les règles pour les prises des pions adverses sont les suivantes:
\begin{itemize}
  \item[\textbullet]Le puit prends la pierre et les ciseaux
  \item[\textbullet]La feuille prend le puit et la pierre
  \item[\textbullet]la pierre prend les ciseaux
  \item[\textbullet]Les ciseaux prennent la feuille
\end{itemize}

\end{itemize}

\section{Déroulement du programme}
 
\end{document}
